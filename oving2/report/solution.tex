\chapter{Solution}
This Chapter will explain the details of the implementation, and our methodology. 
We explain the nescessary components for a pipelined processor, and provide RTL schematics and sequence diagrams for components nescessary to facilitate pipelining.


\section{Methodology}
As this exercise is to improve the single-cycle version of the MIPS-processor. It was natural to use the design from Exercise 1 \cite{ex1report} as a starting point.
The new components needed to realize a pipelined designed were sketched out on paper. We used a bottom up approach, designing the new modules first, and integrated them to the system as they completed.
In order to test our implementation we created some programs specifically designed to provoke different bugs and errors in our design in addition to the more general handout test.

\section{Implementation}
Figure \ref{fig:toplevel} shows all major components in the system. 
The Data memory and Instruction memory was provided in the exercise.
The components that were implemented in Exercise 1, and improved to support multi-cycle operation is as follows:

\begin{description}
  \item[Control] \hfill \\
  The control unit takes the opcode of an instruction, and sets the appropriate control signals. The statefulness that was needed in the single-cycle version has been removed, and the unit is now stateless.  
  \item[Program counter (PC)] \hfill \\
  The program counter keeps track of the address to the next instruction to be executed. If there is a branch, the PC will receive the branch address from the Branch unit. The PC may stall, if there is a control hazard.
  \item[ALU] \hfill \\
  The heart of the processor, the ALU performs an operation issued by the control unit and outputs the result. This module has been updated to support forwarded data.
  \item[Registers] \hfill \\
  The register bank is the short term memory of the architecture, feeding operands to the ALU, and receiving data which may be stored based on control signals.
\end{description}

To make the processor pipelined, we need to introduce a number of new modules.

\begin{description}
  \item[Branch] \hfill \\
  If the instruction is a branch or a jump, this module calculates the address, and passes it forth to the PC module. The difference from the single-cycle version is that this  branch calculation is now performed in a separate unit, instead of beeing a part of the ALU. The advantage is that the branch target is now known in the ID-stage, instead of the EX-stage 
  \item[Forwarder] \hfill \\
  In a pipelined processor, if a instruction is dependant on the preceding instruction's result, we would need to wait for the result to propagate through the pipeline before the current instruction can execute. This is called a Data hazard. Some data hazards can be resolved by forwarding. It is the forwarders task to check if data can be forwarded, and if possible, do so.
  \item[Hazard detection] \hfill \\
  In some cases, data forwarding is not possible. In these cases we need to stall the pipeline. The Hazard detector sends a signal to the PC if a stall is required.
  \item[Pipeline registers] \hfill \\
  Every stage of the pipeline is operating on different instructions. This means that the data and control signals for an instruction must be contained within the current pipeline stage. To achieve this, we insert registers between the stages. To keep the pipeline moving, the data is propagated forward on the rising clock edge. 
\end{description}

\begin{figure}[h!]
    \includegraphics[width=\linewidth]{img/toplel.png}
    \caption{RTL Schematic of the system. Some parts have been left out to avoid clutter.}
    \label{fig:toplevel}
\end{figure}


\subsection{Control}
As illustrated in figure \ref{fig:control}, the control unit performs a simple mapping between instructions and control signals.
The major change in this unit, compared to the one implemented in the previous exercise, is that it is a (mostly) combinatorial circuit, with very little notion of state.
In addition to issuing control signals based on the decoded opcode, the control unit is now also responsible for removing instructions that should not be performed, either because the processor has taken a branch or because it is waiting for one of its operands in the case of load word.
To render an instruction harmless the control unit sets the register write and memory write signals to false, ensuring that the instruction will not alter the state of the program and that the forwarder will not use its result.
In case of a control hazard, the control must remember this in a special register so that it may remove the next instruction. 
In the case of a data hazard it is enough to render the current instruction harmless since it will be issued again on the next cycle, thus no spurious instruction has been issued, only unsatisfiable ones.

\begin{figure}[h!]
    \includegraphics[width=\linewidth]{img/Control.png}
    \caption{RTL schematic of the control unit}
    \label{fig:control}
\end{figure}

\subsection{PC}
The program counter has been simplified slightly compared to exercise 1.
Its only job now, is to increment the counter by one each cycle.
We removed all logic relating to calculating branch targets.
The branching unit (Section \ref{sec:branch}) is responsible for this, the PC simply receives branch targets when a branch is taken.

\subsection{ALU}
The core of the ALU, depicted as "ALU" in Figure \ref{fig:alu}, remains very similar to the implementation of Exercise 1 \cite{ex1report}. It takes two operands, and a control signal vector, named "ALU op". This signal decides what operation that will be performed on the two operands. The result is available on the ALU Result signal vector shortly after.

The rest of Figure \ref{fig:alu} shows the muxing of the possible operands. This addition to the ALU module enable us to use results from further down the pipeline, which has not yet been written to the registers.

A separate unit, the Forwarder (explained in Subsection \ref{section:Forwarder}) calculates if forwarding is possible. The ALU receives two control signals from this unit, the Forward A and Forward B.
Based on these signals, the data used as an operand comes from one of these three sources:
\begin{itemize}
\item ReadData1/2, which is data from the register.
\item MEMData, which is the result from the previous ALU operation (named MEM after the stage it origins).
\item WBData, which is data from the write back stage, that either is data loaded from Data memory or the ALU result from two cycles ago.
\end{itemize}
 
Some instructions requires an immediate value. If this is the case, the ALU Src signal will select the immediate signal over the previously muxed signal.

The final signal we need to mention in this section is the Forward Read Data 2. When a store operation is performed, the ALU is used to calculate the address. The data to be stored may come from any of the sources listed above, and thus the data must be forwarded to the mem stage. Operand B receives the immediate, needed to calculate the address.

\begin{figure}[h!]
    \includegraphics[width=\linewidth]{img/alu.png}
    \caption{RTL schematic of the ALU}
    \label{fig:alu}
\end{figure}

\subsection{Registers}
Unlike a simple processor with no pipelining the registers for a pipelined processor must service two instructions per cycle.
The instruction currently in the ID stage reads data from the registers, while the instruction currently in the WB stage is responsible for writing data.
Apart from this, the registers function exactly as in a single stage multicycle processor, and the only special accomodations required for pipelining is to ensure the control signals come from the correct sources.

\subsection{Branch}
\label{sec:branch}
Detecting branches early is important for a pipelined processor, since each instruction fetched after a branch is potentially wrong.
To detect branches as early as possible we have a branch unit which reads register data as soon as a branch instruction is detected at the ID stage.
When a branch instruction is detected the branch unit calculates the target address and sets the PC address source to read from the branch unit.
Likewise, on a jump instruction the branch unit sends the immediate value of the instruction to the PC.

\subsection{Forwarder}
\label{section:Forwarder}
The forwarder, as described in the ALU section, overrides register reads when more recent data is available further down the pipeline.
The task of the forwarder is to check, for both operands, whether either of the operands reads data from the destination register of an operation further down the pipeline.
First the source register for both of the source registers (or, in case of i-format instructions, the source register) of the operation in the execute stage is compared with the destination register of the operation currently in mem stage. 
If either operand in execute stage reads from the destination register of the operation currently at the mem stage, the forwarder signals that this data is available and should be used in place of the register data.
If an operand is not found as the result in the mem stage, the forwarder checks the WB stage.\\
As with the mem stage, if a source register at the execute stage is similar to a destination register in the wb stage it is selected as operand source by the forwarder. 
This ensures that in the case where the same destination register is used both in the mem stage and the wb stage the most recent one will be chosen.
Note that the forwarder checks forwarding for each source register independently, allowing an instruction to use any combination of reg, mem and wb as its source registers.
In our design this ensures that all dependencies are taken care of except for the case when an instruction is dependant on a result from a load operation which takes an extra cycle before being available at the MEM stage.\\ \\
In addition signal which operands can be forwarded, the forwarder has one more task:
In the case where an instruction in the ID stage uses data from the destination register of the operation in the WB stage the job of the forwarder is to select the write data signal instead of the register read signal.
This is nescessary because the instruction relies on data that cannot be forwarded once it reaches the execute stage, but is not updated in the register file before the instruction that needs it has already read the old value from the register file.

\subsection{Hazard detection}
When the data an instruction depends on can not be forwarded in time we have a data hazard. In our design the only source of data hazards are load words where the next operation reads data from the load word instructions destination register. 
This causes a hazard because it takes one cycle to retrieve memory, which unlike an ALU result does not make the data available when the instruction is in the memory stage. 
To detect these hazards the hazard detector compares the destination register of the operation in the execute stage with the source registers of the operation in ID stage.
When there is a match, and the operation in execute stage is a load operation, the hazard detector issues a data hazard warning, which the rest of the system must respond to accordingly.
In addition to data hazards, when a branch is taken or a jump is performed, it is up to the hazard detector to warn the rest of the system.
When the branch unit sets the pc address source signal to read a branch address, the hazard detector issues a control hazard warning.

\subsection{Pipeline registers}
Apart from the register file, the state of the system is held in pipeline registers. In Figure \ref{fig:toplevel} these are shown as pillars.
Signals which are stored in these pipeline registers are depicted as perforated lines, signifying that it takes a cycle before the signal is driven on the opposite side of the register.
In the pipeline registers we not only want to propagate the execution of an instruction, but also the control data that governs it, shown as the blue modules at the top of each pipeline register.
Making sure execution state is only held in pipeline registers has another benefit, making it easier to deal with hazards that may occur, as described in the next section.

\subsection{Handling hazards}
We have described where hazards occur, and the nescessary components to handling them, but we have not yet stated how.
The two hazards we face in our design are control hazards and data hazards, each being handled separately.

\begin{figure}[h!]
    \includegraphics[width=\linewidth]{img/lw_hazard.png}
    \caption{A sequence diagram showing why we cannot satisfy an instruction dependent on a load word. The red arrow indicates that the forwarder provides invalid data.}
    \label{fig:hazard1}
\end{figure}

\subsubsection{Data hazards}
In the sequence diagram in Figure \ref{fig:hazard1} we show an unhandled hazard. When a load word is issued the following instruction which depends on the destination register cannot read from the data memory output, shown as a red arrow.
Since the dependency cannot be satisfied by the forwarder the add instruction instead reads whatever is output by the datamemory from the previous instruction.
To stop this from happening we must stall the first of the add instructions as shown in Figure \ref{fig:hazard2} 
In order to stall an instruction we stop all the pipeline register upstream of the unsatisfiable instruction from updating. 
By stopping the pipeline registers from updating, the unsatisfiable instruction will be issued again next cycle when it is satisfiable.
In addition to ensuring the instruction is issued again next cycle we make sure the current incarnation of the instruction is rendered harmless.
This is done by setting the control signals regwrite and memwrite to false, which ensures it cannot change the state of the processor.

\begin{figure}[h!]
    \includegraphics[width=\linewidth]{img/lw_hazard_handled.png}
    \caption{When we stall for one cycle we can satisfy the dependant instruction}
    \label{fig:hazard2}
\end{figure}

\subsubsection{Control hazards}
The second hazard we face, occurs when a branch or jump occurs. Figure \ref{fig:ctrl1} shows what happens when a branch is taken.
As soon as a branch is detected at the ID stage the program counter is updated, however by the time the branch is detected a new instruction has already been fetched, shown in red in the figure.
When a control hazard is detected it is therefore the job of the control unit to render the next instruction harmless by inserting a nop in place of the next instruction that arrives.

\begin{figure}[h!]
    \includegraphics[width=\linewidth]{img/ctrl_haz.png}
    \caption{The instruction in red text is fetched before the branch is detected, and should not be performed}
    \label{fig:ctrl1}
\end{figure}
